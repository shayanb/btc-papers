% !TEX root = ../main.tex

\section{Discussion}
... \\
Some setups of the bitcoin wallet clients might not work as they are supposed to in a non-default settings. Specially setups such as Bitcoin-QT and Armory that need to have the access to the complete Blockchain files in local computer. As the blockchain size increases (21 Gigabytes in May 2014), it could be a hard task to keep all the files in the computers hard drive and user might have the need to store them in an external storage such as an external hard. One obstacle for normal user would be that as soon as there is the need to use a setting other than the default one, user has to go through command-line switches run the application. For setups such as Armory in Airgap, most of the non-default settings would fail, such as standalone Bitcoin-QT installation or non-default Bitcoin-QT data directory to store the blockchain file in an external storage. All these would be in usability issues that should be overcome to have a better user-friendly Bitcoin wallet clients. 


\section{A Research Agenda}


% = = = = = = = = = = = = = = = = = = = = = = = = = = = = = = = = = = = = = = = = = =

\begin{itemize}
  \item better import/export of keys
  \item better language and guidance on key management 
  \item how do you know the key is in wallet.dat ?  
  \item How do you know if a file has been read? Specifically, how do you know if your private key file has been read by an app other than your wallet? We have unix \texttt{atime} (time of last access), but we can't log what process read the file. 
  \item publishing public keys. people need to keep these around, and they can use to check if they have the right privkey. 
  \item if you have your pubkey and you have the salt, you can avoid rainbow tables, but your pubkeys get longer. (add salt via commitcoin) 
  \item boyen iterated hashing of passwords -> compute amount of iterations based on money, script
  \item digital death
  \item sinkhole accounts
  \item incentives are there because it's money. not like ssh or ssl. 
  \item min password entropy as a function of how much money its protecting: examine in terms of falling cost to brute force (e.g. w/ EC2), vs rising price of bit coins. -> future proof against lowering EC2 costs and raising bitcoin costs
  \item bitbills, oblivious printing and other physically intuitive media that can behave (sort of) like cash
  \item theft/loss mitigation, making it easier to split wallets to mitigate loss
  \item account rollover vs. password changes
  \item graphical passwords
  \item security of RNGs. i read that vanitygen (for making vanity bitcoin addresses) uses OpenSSL's RNG and it's secure because it's ``used on thousands of websites''. 
  \item air gap -> how good is it?
\end{itemize}

