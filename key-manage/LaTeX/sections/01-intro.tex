% !TEX root = ../main.tex

\section{Introductory Remarks}


% = = = = = = = = = = = = = = = = = = = = = = = = = = = = = = = = = = = = = = = = = =

\begin{itemize}
\item Different paradigm: protecting a key vs protecting a password. 
Bitcoins are secured to their owners by the private key of the address that the bitcoins are stored in. This private keys are not a key or password that the user knows the value so the normal user might not have any ideas about how this works and might lose the private key by simple mistakes

\item Thesis: users aren't ready to manage keys. 
\item State of the art in terms of usability is to somehow convert the key into a password
\item security and deployability (availability) problems
\item keys=something you have. 
\end{itemize}





Due to the meteoritic rise in Bitcoin's exchange rate, users are coming to the realization that it is no longer a crypto plaything; it is actual money. 


We are not arguing that it will hinder adoption, as users will not realize what they are getting into. Rather it will harm traction, since users won't stick around to get fooled twice. 

``Fool me once, shame on\ldots~shame on you. Fool me\ldots~you can't get fooled again.'' --W

Scope: bitcoin can be split into 3 parts: buying bitcoins (involving exchanges, OTC, etc), holding bitcoins (involving wallets, keys, etc), and spending bitcoins (creating transactions, waiting for verifications, etc). this paper focuses on the middle point. 
