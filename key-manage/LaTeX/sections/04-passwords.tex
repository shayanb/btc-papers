% !TEX root = ../main.tex

\section{Managing with a Password}

\cf online banking (cormac herley).

% = = = = = = = = = = = = = = = = = = = = = = = = = = = = = = = = = = = = = = = = = =

\subsection{Password-Derived Key} 

\begin{itemize}
  \item entropy reduction
  \item salt, iteration count
  \item parsing (spec that was followed) requires a tool. if you lose the tool, you may not be able to recover your keys. standard may not be detailed enough, or not properly implemented.
  \item change accounts gets messy
\end{itemize}

% = = = = = = = = = = = = = = = = = = = = = = = = = = = = = = = = = = = = = = = = = =

\subsection{Hosted (Hot Storage)} 
One easier way to manage your bitcoin wallet is by using online wallets. It has it's own advantages and cutbacks. Trusting third party is one issue and the party getting hacked is another. There has been known trustable third parties that just went out of business on their first big hacks, such as instawallet\footnote{\url{http://www.theverge.com/2013/4/3/4180020/bitcoin-service-instawallet-suspended-indefinitely-after-hack}}, Bitcoinica\footnote{\url{http://www.theverge.com/2012/8/10/3233711/second-bitcoin-lawsuit-is-filed-in-california}}, etc. Bitcoinica in 2012 lost more than 60,000 bitcoins due to two successful attacks.\\
In the time of writing this paper the most popular online wallet is blockchain.info\footnote{\url{https://blockchain.info/wallet}} that offers two-factor authentication via email verification, but resistant to forgetting the password.\\ These online services might shutdown due to legal implications or attacks, Also the user might be tricked to reveal his password on a phishing attack\footnote{\url{http://www.theverge.com/2013/4/5/4186808/bitcoin-banker-coinbase-phishing-attacks-user-information-leaked}}, thus these services are not resistant to physical observation.
In short, Hosted (Hot) wallets are not malware resistant nor keep the keys offline, also there is the need of trust and also it is not resistant to physical theft, however because of the usual backup of such services it is resilient to equipment failure. They might offer password reset options to recover the lost password. Compatibility with change keys is as easy as doing a transaction to the new account and they offer immediate access to the funds as they are online services and synced with block chain and usually accessible through a browser thus portable and no need for a new software.

% = = = = = = = = = = = = = = = = = = = = = = = = = = = = = = = = = = = = = = = = = =

\subsection{Hosted (Cold Storage)} 
It is possible to apply the offline storage method to hosted wallets but there are more drawbacks than benefits, not assuming security as the only measure. coinbase announced that more than 87\% of it's users bitcoins are stored in cold storage\footnote{\url {http://blog.coinbase.com/post/33197656699/coinbase-now-storing-87-of-customer-funds-offline}}. It would be secure for most uses but if the total withdrawal in a period of time becomes greater than the funds available in online storage there would be a 48 hours delay on the payments, for scoring purpose we assume this service is fully implemented in cold storage and there is no bitcoins in hot storage of the service.
With keys being offline, it would have resistance toward malwares but no immediate access to the funds, however the user should trust the third party service. The rest of the benefits are the same as hosted on online storage.

%\begin{itemize}
%  \item Offline wallets aka cold storage. not immediately available, transactions queueing up (coinbase). 
% \end{itemize}

