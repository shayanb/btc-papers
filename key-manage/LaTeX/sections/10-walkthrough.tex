% !TEX root = ../main.tex

\section{Evaluation Methodology}

%intro from TOR paper -> undone citations
Another method to evaluate the usability of these techniques is heuristic evaluation (citation). In this paper, we employ cognitive walkthrough as our methodology for usability evaluation.
... 

% = = = = = = = = = = = = = = = = = = = = = = = = = = = = = = = = = = = = = = = = = =

\subsection {The Core Tasks}
The core tasks that we are going to perform for each key management technique are as follows.

\begin{itemize}
\renewcommand{\labelitemi}{\bf CT-1} \label{sec:ct-1}
\item Finalize a receiving address and balance from the primary device\footnote{primary device is the initial device that the key is generated on} 
\renewcommand{\labelitemi}{\bf CT-2} \label{sec:ct-2}
\item Authorize the transaction from the primary device 
\renewcommand{\labelitemi}{\bf CT-3} \label{sec:ct-3}
\item Authorize the transaction from the secondary device\footnote{any device other than the primary device}
\renewcommand{\labelitemi}{\bf CT-4} \label{sec:ct-4}
\item Losing the main credential (recovery options)
\end{itemize}

The core tasks will be performed with the following clients with default configuration on each category:

\begin{enumerate}
\item Key in file (default):  \bitcoinclient 
\item Password Protected: \multibit 
\item Air gap: Bitcoin Armory \footnote{\url{https://bitcoinarmory.com/}}
\item Offline Storage: Paper Wallet
\item Password-drived key: Brain Wallet
\item Hosted: Blockchain.info wallet

\end{enumerate}

% = = = = = = = = = = = = = = = = = = = = = = = = = = = = = = = = = = = = = = = = = =

\subsection{Usability Guidelines} %copied from TOR paper with some small modifications!

	The set of guidelines that we use to evaluate each of the core tasks are as follows.
	
\begin{description}
	\item[G1] Users should be aware of the steps they have to perform to complete a core task.
	\item[G2] Users should be able to determine how to perform these steps.
	\item[G3] Users should know when they have successfully completed a core task.
	\item[G4] Users should be able to recognize, diagnose, and recover from non-critical errors.
	\item[G5] Users should not make dangerous errors from which they cannot recover.
	\item[G6] Users should be comfortable with the terminology used in any interface dialogues or documentation.
	\item[G7] Users should be sufficiently comfortable with the interface to continue using it.
	\item[G8] Users should be aware of the application's status at all times.
\end{description}

	These guidelines are drawn from a variety of sources \cite{cw,johnny,karat,p3p,pvo,clark} and are intended for evaluating Bitcon Key management specifically. However they are suitably broad and may find application in other usable privacy walkthroughs. We now individually justify the inclusion of each.
	
\subsubsection*{G1: Users should be aware of the steps they have to perform to complete a core task.} This is a restatement of the first guideline of Whitten and Tygar \cite{johnny}. Every user of a new application knows certain things before using the system and learns certain things during the use of the system. In the cognitive walkthroughs we carry out here, the presupposition is that the user knows enough to start the process for each core task---in the case of installation, the user can download the installation file and open it; in the case of configuration, the user can explore the user interface or follow cues. We are evaluating how the application cues the user to perform the intermediary steps between these broadly defined tasks. 
	
\subsubsection*{G2: Users should be able to determine how to perform these steps.} Once the user is aware of what intermediary steps are necessary, she must be able to figure out how to perform these steps. This is the second guideline in \cite{johnny}. It is assumed the user has a mental model of how the system works. It is thus important that the system model be harmonized with the user's mental model if the user is to be successful in performing the necessary steps required to complete each core task  \cite{cw}. What is less obvious is why we cannot fully rely on the user to simply modify her mental model when given conflicting information.

A predominate reason is that humans have a stronger preference for confirming evidence than disconfirming evidence when evaluating their own hypotheses. This cognitive bias is well illustrated by Wason \cite{bias1}, who conducted a study where a set of subjects were given the following sequence of numbers: 2,4,6. The subjects were told that the numbers followed a rule, and their task was to determine the rule by proposing their own sequence of numbers, which would be declared as matching the rule or not. The rule was any ascending sequence of numbers. However most subjects derived a more complicated rule, such as every ascending sequence of numbers differing by two. The point of this test was that the subjects, on average, had a preconceived idea of what the rule was and only proposed sequences to confirm that rule, instead of also proposing sequences that would falsify their perceived rule.

Confirmation bias is important in usability because it proposes that users are biased toward only seeking and accepting information that confirms their mental model, and thus may avoid or even ignore information that contradicts it. It cannot reasonably be expected that users will easily and quickly adapt their mental model to new information.

A second concern with how users perform these steps is that security is a secondary goal \cite{johnny,karat}. If the user is given two paths to completing a core task---one that is fast but not secure, and one that is slower but more secure--it cannot be assumed that the user will take the latter approach. In fact, studies in behavioural economics demonstrate that humans often prefer smaller immediate payoffs to larger future payoffs, such as \$50 today instead of \$100 a year from today \cite{bias2}. Using software securely has a greater (usually non-monetary) payoff for the user, but this utility has to be substantially higher than the alternative to justify the delay in achieving it.

\subsubsection*{G3: Users should know when they have successfully completed a core task.} In other words, users should be provided with ample feedback during the task to ensure they are aware of its successful completion. This principle has been proposed in the context of heuristic evaluation \cite{cw} and for a cognitive walkthrough \cite{pvo}. It was also mentioned by Cranor \cite{p3p}. In Bitcoins, it is essential that the user is provided with confirmation of the task's finalization, such as successful back up of \walletfile.

\subsubsection*{G4: Users should be able to recognize, diagnose, and recover from non-critical errors.} Users will likely make errors in performing the core tasks and it is important for them to be able to recover from these errors \cite{cw}. It is important for users to be given concise error messages.

\subsubsection*{G5: Users should not make dangerous errors from which they cannot recover.} This guideline is from Whitten and Tygar \cite{johnny}. In Bitcoin subject, the most dangerous error is to reveal the private key which is associated with the address that holds the funds. Also in case of backups, the corrupted \walletfile would be useless for recovery.

\subsubsection*{G6: Users should be comfortable with the language used in any interface dialogues or documentation.} Wharton \textit{et al.} emphasize that applications should use simple, natural, and familiar language \cite{cw}. 

\subsubsection*{G7: Users should be comfortable with the interface.} This is the fourth principle of usable security of Whitten and Tygar \cite{johnny}, and is an essential part of the principal of psychological acceptability quoted by Bishop \cite{pa}. 


\subsubsection*{G8: Users should be aware of the system status at all times.} This principle was proposed in the context of heuristic evaluation \cite{cw} and cognitive walkthrough \cite{pvo}. Cranor advocates the use of `persistent indicators' that allow the user to see all the required information at a glance \cite{p3p}. In terms of Bitcoin, we are looking for indicators that show the balance and the addresses that is included in the \walletfile and also the transaction history.

%===========================================================
\subsubsection{Key in file}
We assume the user knows how to download the client from the main site. It has a straight forward wizard installation procedure, user runs the application for the first time. %it should be online to show to final balance

% Section 2: before the walkthrough: There could be this mindset that there are steps needed to sign up for bitcoin or to register on the site, however this is not the case.

 \paragraph{CT1}  On the first run user would see the "Overview" page that might be confusing for the first time user because it is just showing "Out of Sync" in red. However the first task that is to have a bitcoin address is already done (CT1), and he can find the address with clicking on the "Receive Coins" tab of the application that might be confused with the "Addresses" tab (that is like a contact list for other addresses), this violates G2 and G3.
User should be aware that the application must be connected to internet to get synced (G1), however except small status indicator on the bottomright side of the window that shows a small red cross in-between two black windows, there is no other alerts and that is a violation of G1, even with mouse over on the icon it would say "0 active connections to bitcoin network" that for the user not familiar with the terminology, does not reflect the meaning that the client should be online(G4 and G6).\ 
For the Balance he would not see the final balance until the sync is done (This is more than 10 GB of download and because of the peer to peer nature of the download it may take days to be synced \footnote {\url { https://en.bitcoin.it/wiki/Satoshi_Client_Block_Exchange\#Performance}} ).
\paragraph{CT2} This is the easiest task for this client as all the keys are already loaded in the application from \walletfile and user can use the "Send" tab to authorize a transaction to a given bitcoin address.
%the minimum knowledge of bitcoin addresses is needed, to not to make any mistakes. for example if the receiving address is in the wrong format, there is no error to show that is so but just a red background appears on the address which is a slight violation of G4. \

\paragraph{CT3} By installing \bitcoinclient on the new device, a new wallet file is generated. In order to complete CT3, user might try finding import options in the newly installed wallet client or drag and dropping the file into the client, but failed to do so because there is no such an option, even if the user tries to find the documentation he would fail as there is nothing in the help menu except debug window that is for advance user to tweak the application.Most of the users would stop here, However there is one way to do so and that is to replace his newly generated wallet file with his previously owned wallet file, the one that was setup in CT1, to have access to his funds, but nothing has been mentioned about this in the documentations (G1). To do so user have to back up the \walletfile with the "backup wallet..." option in the "File" tab and chose a directory to save the \walletfile. \

Now he need to have a secure way to transfer this file to the secondary device, which could be a USB flash drive. Depending on the nature of primary and secondary devices and also the wallet clients the method of copying the wallet file might differ \footnote{A Comparison of Secure File Transfer Mechanisms \url{http://www.process.com/tcpip/sft.pdf}}

One more obstacle is to find the path to copy \walletfile on the new device since there is no import option on the \bitcoinclient, this would be possible by either searching the local file system for \walletfile, which he might not succeed due to non-searchable system reserved folders or not knowing the exact file name, or searching on the internet for the answer. On Mac OS X, the path is /Users/User/Library/Application Support/Bitcoin/wallet.dat \footnote{\url{https://en.bitcoin.it/wiki/Data_directory}}. The next step is to replace the new \walletfile with the one from the primary device. it should be noted that the name of the file should be exactly \walletfile for the \bitcoinclient to be able to read the file. Some of errors that the user might encounter during this procedure are as follows:
\begin{itemize}
	\item User might accidentally copy \walletfile from the primary device wallet client path instead of the one he backed up, this would cause to have a corrupted \walletfile and not readable by the secondary device's \bitcoinclient, \bitcoinclient has a procedure to lock \walletfile while it is in use, this may cause to have a corrupted file if the file has been copied while it was locked %redescribe this one ! 
	\item There is the possibility to replace the backed up \walletfile instead of the newly generated wallet file and lose the backup file on the secondary device (G4)
	\item On the secondary device the final balance might be wrong and there would be the need to resynchronize the blockchain to have the correct final balance (G3)
\end{itemize}
As for CT3, all guidelines (G1-G8) has been violated.\

\paragraph{CT4} For recovery options, the procedure is the same as CT3 but on the same device. It should be noted that the saved backup \walletfile should be kept in a secure storage(*** NEED REFERENCE FOR SECURE STORAGE****) as it contains all the keys and so all the funds stored in the addresses. User might try to store the backed up file on the same device that could lead to losing all his funds due to hardware failure (G5). %backup -> theft/losing


\subsubsection{Password Protected}
Although it is possible to encrypt the \walletfile with \bitcoinclient, there is no emphasize or alerts to do so, however in \multibit client one of the recommended first steps is to password protect the wallet file, this is one of the reasons that this client has been chosen to be analyses for this cognitive walkthrough, also because it uses SPV for faster blockchain synchronization this client is more popular amongst new users.

\paragraph{CT1}On the first run the welcome page would pop up that has the explanation of the core tasks that could be done with \multibit such as where the send, request and transaction tabs are and how to password protect the wallet file and also help options for all the other functionalities (G1,G2). 

\multibit help option gives direct and non-technical guides on how to do the desired functions. The interface is pretty easy to understand and it shows the status of the program (Online, offline, out of sync) on the bottom left  status bar, the balance of the user's wallet on the up left and the latest price of bitcoin on the up right of the window(G8). There are not too many jargons and technical vocabulary used (G6). After the first run the receiving address in finalized and it is possible to receive bitcoins (CT1). By clicking on "Request" tab the finalized address and it's relative QR Code\footnote{\url{http://en.wikipedia.org/wiki/QR_code}} appears, User can now enter his desired amount to be received to generate the appropriate QR code with his input value, even though just having the address suffice to complete this task. On the section titled "Your receiving addresses" all the addresses stored in the wallet file is shown, By clicking on the "New" button he can generate a new bitcoin addresses. \multibit also has the short come of not showing the user that he has to be connected to internet to get synced and it will stay on "connecting" mode,this slightly violates G1 and G2.
As it is mentioned on the "Welcome page", every option in \multibit would show help tips with hovering the mouse over the option.

\paragraph{CT2} To authorize a transaction user has to click on the "send" tab, \multibit has a really simple and complete interface to do so(G7), where it is possible to import the sending address by QR code or from the clipboard or just by typing it in the "Address" field. If the address in not correctly formatted or the amount of the transaction is more than the balance, a fully detailed error message would pop up to explain what went wrong (G4). Also if the wallet client is not synced, the send button would be disabled. If it's synced and there is no error on the balance and the destination address, the transaction is complete after clicking on send button and approving the transaction.


\paragraph{CT3} Same as the "Key in file" this task is not a main functionality of \multibit. On the primary device user has to look in the options to find the backup options, here it is in Tools - "Export Private Keys" (violates G6). It will show details about current wallet file, the path for the export file to be saved and also password of export file that is enabled by default. In case user tries to save the exported file without password, there would be a warning saying "Anyone who can read your export file can spend your bitcoin." in red. By having a password protected export file, use can copy the file safer to the secondary device. By clicking on "Export Private Keys" button it will save the wallet file in the given path and also checks if the file is readable (no violation of G4 and G5). On the secondary device user installs a fresh copy of \multibit and looks for Import options that is Tools - Import Private Keys. The window looks the same as the Export window but with import functionality. User has to browse for his exported file from the primary device and type in the password and then click on "Import Private Keys" button. It will confirm the completion of the import and changes the balance according to the balance of the new imported addresses. Now he's able to authorize a transaction to any given destination addresses.


\paragraph{CT4} This task is the same as CT3 but on the same device. The password-protected back up file should be stored in a secure storage that would prevent it from being lost or stolen, however it would be hard to recover the wallet file without knowing its password. 

%Analysis of the core tasks for the this section was done on \multibit that one of the most used bitcoin clients among new users due to usage of SPV protocol and less time to sync the blockchain to show the updated amount of funds. CT1 to CT3 are the same as "Key in file" technique, however for CT4, whether user has forgot his passphrase on the encrypted wallet or loses his wallet file, it is unlikely to recover the keys.


\subsubsection{Air gap}
\label{sec:air gap}


\subsubsection{Offline Storage}
There are different options for Offline Storage of a bitcoin as described in Section \ref{offline storage}. For the cognitive walkthrough we choose the paper wallet to include a different approach of saving the keys as the other methods has similar parts to key in file or the Air gap technique.\\
For CT1 user could enter any of the bitcoin paper wallet sites (e.g www.bitaddress.org). By mouse movement or adding some text to the proposed fields, user can make a random seed to generate a random unpredictable address. The address would appear on the left of the screen with the private key on the right labeled as secret. As for CT2 and CT3 there is no difference in the primary device and the secondary as the key should be imported from the paper wallet into the wallet client. The best way to do so is to have an online wallet on blockchain.info or use the Bitcoin Armory application to import the private key or use the signing transaction option to authorize the transaction. On loosing the key (CT4) there is no way to recover the keys without the use of any other backups.

\subsubsection{Password-drived key}
To start using this technique user has to either run the offline brain wallet or use the online services such as brainwallet.org. On entering the site user is asked to enter a customized personal string to produce the random seed to generate the bitcoin address (CT1). For example with the string "brainwallet strong password" the address of \\ "12HH5MmDhPsZWLb1QZKhptzGpPG1R73gwy" has been generated. note that the string should not be shared or the funds would be available to whoever that has the string. Same as the offline storage on any system that has access to the software it is possible to produce the keys from the users string so CT2 and CT3 are the same and easy to do on any system. On forgetting the credentials, it would be impossible to recover the keys (CT4).


\subsubsection{Hosted}
Hosted services are the easiest ones to use for the new users. There would be a sign up for an online wallet that we chose blockchain.info \footnote{\url{https://blockchain.info/wallet}} as it is the most famous online wallet on the time of the writing. The user clicks on "Start a new wallet" and provides an email address with a password for his login. On the first page it has a warning that on forgetting the password there is no way to recover the account (CT4), however it gives the user a "Wallet Recovery Mnemonic" that is a set of words that could be used to recover the account. On the first login the user would have a finalized address that he can use to receive and send bitcoins. The interface is pretty user-friendly and easy to use. As for CT2 and CT3 there is no difference between the devices as a simple web browser would give access to the account for the user.
