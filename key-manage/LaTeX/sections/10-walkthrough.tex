% !TEX root = ../main.tex

\section{Evaluation Methodology}

%intro from TOR paper -> undone citations
Another method to evaluate the usability of these techniques is heuristic evaluation (citation). In this paper, we employ cognitive walkthrough as our methodology for usability evaluation.
... 

% = = = = = = = = = = = = = = = = = = = = = = = = = = = = = = = = = = = = = = = = = =

\subsection {The Core Tasks}
The core tasks that we are going to perform for each key management technique are as follows.

\begin{itemize}
\renewcommand{\labelitemi}{\bf CT-1} \label{sec:ct-1}
\item Finalize a receiving address and balance from the primary device\footnote{primary device is the initial device that the key is generated on} 
\renewcommand{\labelitemi}{\bf CT-2} \label{sec:ct-2}
\item Authorize the transaction from the primary device 
\renewcommand{\labelitemi}{\bf CT-3} \label{sec:ct-3}
\item Authorize the transaction from the secondary device\footnote{any device other than the primary device}
\renewcommand{\labelitemi}{\bf CT-4} \label{sec:ct-4}
\item Losing the main credential (recovery options)
\end{itemize}

The core tasks will be performed with the following clients with default configuration on each category:

\begin{enumerate}
\item Key in file (default):  \bitcoinclient 
\item Password Protected: \multibit 
\item Air gap: Bitcoin Armory \footnote{\url{https://bitcoinarmory.com/}}
\item Offline Storage: Paper Wallet
\item Password-drived key: Brain Wallet
\item Hosted: Blockchain.info wallet

\end{enumerate}

% = = = = = = = = = = = = = = = = = = = = = = = = = = = = = = = = = = = = = = = = = =

\subsection{Usability Guidelines} %copied from TOR paper with some small modifications!

	The set of guidelines that we use to evaluate each of the core tasks are as follows.
	
\begin{description}
	\item[G1] Users should be aware of the steps they have to perform to complete a core task.
	\item[G2] Users should be able to determine how to perform these steps.
	\item[G3] Users should know when they have successfully completed a core task.
	\item[G4] Users should be able to recognize, diagnose, and recover from non-critical errors.
	\item[G5] Users should not make dangerous errors from which they cannot recover.
	\item[G6] Users should be comfortable with the terminology used in any interface dialogues or documentation.
	\item[G7] Users should be sufficiently comfortable with the interface to continue using it.
	\item[G8] Users should be aware of the application's status at all times.
\end{description}

	These guidelines are drawn from a variety of sources \cite{cw,johnny,karat,p3p,pvo,clark} and are intended for evaluating Bitcon Key management specifically. However they are suitably broad and may find application in other usable privacy walkthroughs. We now individually justify the inclusion of each.
	
%===========================================================


\subsubsection*{G1: Users should be aware of the steps they have to perform to complete a core task.} This is a restatement of the first guideline of Whitten and Tygar \cite{johnny}. Every user of a new application knows certain things before using the system and learns certain things during the use of the system. In the cognitive walkthroughs we carry out here, the presupposition is that the user knows enough to start the process for each core task---in the case of installation, the user can download the installation file and open it; in the case of configuration, the user can explore the user interface or follow cues. We are evaluating how the application cues the user to perform the intermediary steps between these broadly defined tasks. 
	
\subsubsection*{G2: Users should be able to determine how to perform these steps.} Once the user is aware of what intermediary steps are necessary, she must be able to figure out how to perform these steps. This is the second guideline in \cite{johnny}. It is assumed the user has a mental model of how the system works. It is thus important that the system model be harmonized with the user's mental model if the user is to be successful in performing the necessary steps required to complete each core task  \cite{cw}. What is less obvious is why we cannot fully rely on the user to simply modify her mental model when given conflicting information.

A predominate reason is that humans have a stronger preference for confirming evidence than disconfirming evidence when evaluating their own hypotheses. This cognitive bias is well illustrated by Wason \cite{bias1}, who conducted a study where a set of subjects were given the following sequence of numbers: 2,4,6. The subjects were told that the numbers followed a rule, and their task was to determine the rule by proposing their own sequence of numbers, which would be declared as matching the rule or not. The rule was any ascending sequence of numbers. However most subjects derived a more complicated rule, such as every ascending sequence of numbers differing by two. The point of this test was that the subjects, on average, had a preconceived idea of what the rule was and only proposed sequences to confirm that rule, instead of also proposing sequences that would falsify their perceived rule.

Confirmation bias is important in usability because it proposes that users are biased toward only seeking and accepting information that confirms their mental model, and thus may avoid or even ignore information that contradicts it. It cannot reasonably be expected that users will easily and quickly adapt their mental model to new information.

A second concern with how users perform these steps is that security is a secondary goal \cite{johnny,karat}. If the user is given two paths to completing a core task---one that is fast but not secure, and one that is slower but more secure--it cannot be assumed that the user will take the latter approach. In fact, studies in behavioural economics demonstrate that humans often prefer smaller immediate payoffs to larger future payoffs, such as \$50 today instead of \$100 a year from today \cite{bias2}. Using software securely has a greater (usually non-monetary) payoff for the user, but this utility has to be substantially higher than the alternative to justify the delay in achieving it.

\subsubsection*{G3: Users should know when they have successfully completed a core task.} In other words, users should be provided with ample feedback during the task to ensure they are aware of its successful completion. This principle has been proposed in the context of heuristic evaluation \cite{cw} and for a cognitive walkthrough \cite{pvo}. It was also mentioned by Cranor \cite{p3p}. In Bitcoins, it is essential that the user is provided with confirmation of the task's finalization, such as successful back up of \walletfile.

\subsubsection*{G4: Users should be able to recognize, diagnose, and recover from non-critical errors.} Users will likely make errors in performing the core tasks and it is important for them to be able to recover from these errors \cite{cw}. It is important for users to be given concise error messages.

\subsubsection*{G5: Users should not make dangerous errors from which they cannot recover.} This guideline is from Whitten and Tygar \cite{johnny}. In Bitcoin subject, the most dangerous error is to reveal the private key which is associated with the address that holds the funds. Also in case of backups, the corrupted \walletfile would be useless for recovery.

\subsubsection*{G6: Users should be comfortable with the language used in any interface dialogues or documentation.} Wharton \textit{et al.} emphasize that applications should use simple, natural, and familiar language \cite{cw}. 

\subsubsection*{G7: Users should be comfortable with the interface.} This is the fourth principle of usable security of Whitten and Tygar \cite{johnny}, and is an essential part of the principal of psychological acceptability quoted by Bishop \cite{pa}. 


\subsubsection*{G8: Users should be aware of the system status at all times.} This principle was proposed in the context of heuristic evaluation \cite{cw} and cognitive walkthrough \cite{pvo}. Cranor advocates the use of `persistent indicators' that allow the user to see all the required information at a glance \cite{p3p}. In terms of Bitcoin, we are looking for indicators that show the balance and the addresses that is included in the \walletfile and also the transaction history.

%===========================================================
\subsubsection{Key in file}
We assume the user knows how to download the client from the main site. It has a straight forward wizard installation procedure, user runs the application for the first time. 

 \paragraph{CT1}  On the first run user would see the "Overview" page that might be confusing for the first time user because it is only indicating "Out of Sync" in red. However the first task that is to have a bitcoin address is already done (CT1), and he can find the address with clicking on the "Receive Coins" tab of the application that also could be confused with the "Addresses" tab (Addresses tab is similar to a contact list of other addresses), this violates G2 and G3.
User should be aware that the application must be connected to internet to get synced, however except small status indicator on the bottomright side of the window that shows a small red cross in-between two black windows, there is no other alerts and that is a violation of G1, even with mouse over on the icon it would say "0 active connections to bitcoin network" that for the user not familiar with the terminology, does not reflect the meaning that the client should be online(G4 and G6).\ 
For the Balance he could not see the final balance until the sync is done (This is more than 17 GB of download and because of the peer to peer nature of the download it may take days to be synced \footnote {\url { https://en.bitcoin.it/wiki/Satoshi_Client_Block_Exchange\#Performance}} ).

\paragraph{CT2} This is the easiest task for this client as all the keys are already loaded in the application from \walletfile and user can use the "Send" tab to authorize a transaction to a given bitcoin address.
%Remove this? -> the minimum knowledge of bitcoin addresses is needed, to not to make any mistakes. for example if the receiving address is in the wrong format, there is no error to show that is so but just a red background appears on the address which is a slight violation of G4. \

\paragraph{CT3} By installing \bitcoinclient on the new device, a new wallet file is generated. In order to complete CT3, user might try looking for import options in the newly installed wallet client or drag and dropping the \walletfile into the client, but failing to do so because there is no such an option, even if the user tries to find the documentation he would fail as there is nothing in the help menu except debug window that is for advance user to tweak the application. Most of the users would stop here, However there is one way to do so and that is to replace his newly generated \walletfile with his previously owned \walletfile from the primary device, the one that was setup in CT1, to have access to his funds, but nothing has been mentioned about this in the documentations (G1). To do so user have to back up the \walletfile with the "backup wallet..." option in the "File" tab and chose a directory to save the \walletfile. \

Now he need to have a secure way to transfer this file to the secondary device, which could be a USB flash drive. Depending on the nature of primary and secondary devices and also the wallet clients the method of copying the wallet file might differ. %refrence to secure file transfer?

One more obstacle is to find the path to copy \walletfile on the new device since there is no import option on the \bitcoinclient, this could be possible by either searching the local file system for \walletfile, which he might not succeed due to non-searchable system reserved folders or not knowing the exact file name, another way is search on the internet for the answer. On Mac OS X, the path is /Users/User/Library/Application Support/Bitcoin/wallet.dat \footnote{\url{https://en.bitcoin.it/wiki/Data_directory}}. The next step is to replace the new \walletfile with the one from the primary device. it should be noted that the name of the file should be exactly \walletfile for the \bitcoinclient to be able to read the file. Some of errors that the user might encounter during this procedure are as follows:
\begin{itemize}
	\item User might accidentally copy \walletfile from the primary device wallet client path instead of the one he backed up, this would cause to have a corrupted \walletfile and not readable by the secondary device's \bitcoinclient, \bitcoinclient has a procedure to lock \walletfile while it is in use, this may cause to have a corrupted file if the file has been copied while it was locked % redescribe this one? remove it? 
	\item There is the possibility to replace the backed up \walletfile instead of the newly generated wallet file and lose the backup file on the secondary device (G4)
	\item On the secondary device the final balance might be wrong and there would be the need to resynchronize and rescan the blockchain to have the correct final balance (G3)
\end{itemize}
As for CT3, all guidelines (G1-G8) has been violated.\

\paragraph{CT4} For recovery options, the procedure is the same as CT3 but on the same device. It should be noted that the backed up \walletfile should be kept in a secure storage %refrence for secure storage?
as it contains all the keys and so all the funds stored in the addresses. User might try to store the backed up file on the same device that could lead to losing all his funds due to hardware failure (G5).

%===========================================================

\subsubsection{Password Protected}
Although it is possible to encrypt the \walletfile in \bitcoinclient with a password, there is no emphasize or alerts to do so, however in \multibit client one of the recommended first steps is to password protect the wallet file, this is one of the reasons that this client has been chosen to be analyses for this cognitive walkthrough, also because it uses SPV for faster blockchain synchronization this client is more popular amongst new users.

\paragraph{CT1}On the first run the welcome page pops up that has the explanation of the core tasks that could be done with \multibit such as where the send, request and transaction tabs are and how to password protect the wallet file and also help options for all the other functionalities (Good use of G1 and G2). 

\multibit help option gives direct and non-technical guides on how to do the desired functions. The interface is pretty easy to understand and it shows the status of the program (Online, offline or Out of Sync) on the bottom left  status bar, the balance of the user's wallet on the up left and the latest price of bitcoin on the up right of the window(Complies G8). There are not too many jargons and technical vocabulary used (Complies G6). After the first run the receiving address is finalized with the updated final balance and it is possible to receive bitcoins (CT1). By clicking on "Request" tab the finalized address and it's relative QR Code\footnote{\url{http://en.wikipedia.org/wiki/QR_code}} appears, User can now enter his desired amount to be received to generate the appropriate QR code with his input value, even though just having the address suffice to complete this task. On the frame titled "Your receiving addresses" all the addresses stored in the wallet file is shown, By clicking on the "New" button user can generate a new bitcoin addresses. \multibit also has the short come of not showing the user that he has to be connected to internet to get synced and it will stay on "connecting" mode,this slightly violates G1 and G2.
As it is mentioned on the "Welcome page" of the application, every option in \multibit has the ability to show help tips by hovering the mouse over that option.

\paragraph{CT2} To authorize a transaction user has to click on the "send" tab, \multibit has a really simple and complete interface to do so(Good use of G7), where it is possible to import the sending address by QR code or from the clipboard or just by typing it in the "Address" field. If the address in not correctly formatted or the amount of the transaction is more than the balance, a fully detailed error message pops up to explain what went wrong (Complies G4). Also if the wallet client is not synced, the send button would be disabled. If it's synced and there is no error on the balance and the destination address, the transaction is complete after clicking on send button and approving the transaction.

\paragraph{CT3} Same as the "Key in file" this task is not one of the main functionalities of \multibit. On the primary device user has to look in the options to find the backup options, It could be found in Tools - "Export Private Keys" (violates G6). It will show details about current wallet file, the path for the export file to be saved and also a password for the exported file that is enabled by default. In case user tries to save the exported file without password, there would be a warning saying "Anyone who can read your export file can spend your bitcoin." in red. By having a password protected export file, user can copy the file safer to the secondary device. By clicking on "Export Private Keys" button it will save the wallet file in the given path and also checks if the file is readable (no violation of G4 and G5). On the secondary device user installs a fresh copy of \multibit and looks for Import options that is under Tools - Import Private Keys. The window looks the same as the Export window but with import functionality. User has to browse for his exported file from the primary device and type in the password and then click on "Import Private Keys" button. It will confirm the completion of the import and changes the balance according to the balance of the new imported addresses. Now user is able to authorize a transaction to any given destination addresses.

\paragraph{CT4} This task is the same as CT3 but on the same device. The password-protected back up file should be stored in a secure storage that prevents it from being lost or stolen, however it would be hard to recover the wallet file without knowing its password. 

%===========================================================

\subsubsection{Air gap}
\label{air gap}
This technique has not yet been documented well and it is suitable for advanced users only. For the cognitive walkthrough, we assume user has good knowledge of bitcoin terms (G6) and how the air gap method works (G1). Bitcoin Armory is one of the advanced bitcoin wallets that is going to be used for this walkthrough, although for the online system it should be executed while \bitcoinclient is already running, to be able to load the blockchain. It is possible to use some other applications to implement the airgap, however this is by far the most secure way to implement air gap for bitcoin wallet.

\paragraph{CT1} 
First, User should install Bitcoin Armory on the offline computer. On the start, the welcome page offers the option to "Import Existing Wallet" and "Create Your First Wallet!". User creates his wallet, passphrase is a mandatory option, Armory also asks for the third time verification of the passphrase and warns the user not to forget his passphrase (appropriate guidelines for G5). After this step, a backup window pops up with the options to print a paper wallet or save a digital backup of the wallet, also warns the user if he decides not to backup his wallet (Complies G5). Now the Bitcoin address is generated, however in order to see the final balance of the account, it needs to be connected to \bitcoinclient and \bitcoinclient must be online and synced, that should not be done on the offline computer. User with less knowledge of air gap technique might click on the "Offline Transaction" button that offers a short documentation of the steps to be taken to sign a transaction. One point that has not been mentioned in the documentation is that user must click on "Receive Bitcoins" to generate the bitcoin address in the wallet file (slightly violates G3,G4). On wallet window, there is the option to "Create Watching-Only Copy". This option is being used to copy the bitcoin addresses, not the private keys, for the online computer to display the balances. This option is used to create the file to be copied to the online computer.\\
Now user should install Bitcoin Armory and \bitcoinclient on the online computer to have access to the updated balances of his accounts. This time on the first run, user should click on "Import Existing Wallet" and chose to import the digital backup or watch-only wallet, Then chose the watch-only back up file that has been copied from the offline computer. After the application is done with syncing with the blockchain and scanning it for transactions that contains the imported bitcoin address, the balance is shown and the user has finalized his receiving address.


\paragraph{CT2, CT3} 
In Airgap technique, There is no primary or secondary device in the sense that we have defined these terms. It's the offline computer that should be kept secure and disconnected from the internet, and there is the other device that should be online. Thus, the same steps (CT1) are required on any given new online device, to add the watch-only wallet and wait for it to synchronize with the blockchain.\\
Now on the online device with the watch-only wallet added, user should click on "Offline Transactions" on the main window. It should be noted that unless the wallet client is online, the "Create New Offline Transaction" is disabled. Now by clicking on this option, user will be asked to enter the receiving addresses, amounts and any comment on the transaction. There should be enough balance in the sending account for the offline transaction option in Armory to generate the transaction data. If everything is fine with the configuration, by clicking on "Continue" the unsigned transaction data is generated. User can now save these data on a file in a usb stick and should transfer this data to the offline computer. Armory has good documentation of the needed steps on each window (good use of G1). As also mentioned in this step's documentation, these unsigned transaction data has no private data and no harm can be done by the attacker whom captures this file, but only privacy issues that these data includes the sending address.

Next step is on the offline computer to sign the transaction. By clicking on "Offline Transaction" and then "Sign Offline Transaction" , there is now the option to load the unsigned transaction data file. Armory would alarm the user to review the transaction information such as the amount and the receiving addresses in each step (Good behaviour for G5). By clicking on "sign" button, it will show the transaction details and waits for user's approval.  Next it will show the signed transaction data and user can save it in the file. It should be noted that on the side of the window it will show if the shown data is a signed or unsigned transaction (follows G4). 
Once again this file should be transferred to the online computer and get loaded in the offline transaction window. Now the difference is that the "sign" button is disabled but the "broadcast" button has been enabled. By clicking on "broadcast" it will once again show the detail of the transaction for the user to review(Complies G5). This would be the last window for the transaction and by clicking on "continue" button it will broadcast the transaction to the bitcoin network and the bitcoins will be sent to the receiving address.

\paragraph{CT4} 
Armory gives lots of options and opportunities for the user to back up his keys. In the time of creating the wallet, there are multiple windows and alerts convincing user to back up his wallet, either in digital format or paper wallet even if user decides not to back up his wallet at the start, he would have the chance to do so afterwards by going to his wallet and clicking on "Backup This Wallet". On the backup window, There are options to back up a digital copy or paper copy and also more options that are mostly more secure ways to backup the wallet and would fall outside the scope of this paper. By clicking on the "Make Paper Backup", The paper backup is shown to the user containing a Root key that consist of 18 sets of 4-characters words and a QR code with the ability to print the page. To restore the paper wallet backup, on the main page, user can click on "Import or Restore Wallet" and select "Single-sheet Backup" option, now he will be asked to input Root Key from the paper wallet, and on confirming the Root Key the wallet would be restored and added to the Armory wallet client. That being said, there is the "Digital Backup" option that would backup an unencrypted version of the wallet file and should be kept secure. The digital backup is an easier way to backup and recover as it the recovery would be just importing the file in the "Import or Restore Wallet" window by selecting "Import Digital Backup or watch-only wallet" as we did for the online computer in CT1. Armory also gives this option to do the import procedure for testing the backups to see if there is no error in the backup file.

%===========================================================


\subsubsection{Offline Storage}
There are different methods to use for Offline Storage of a bitcoin as described in Section \ref{offline storage}. For the cognitive walkthrough we choose the paper wallet to include a different approach of saving the keys as the other methods has similar parts to Air gap technique. Bitaddress \footnote{\url{https://www.bitaddress.org}} is a well-known Bitcoin paper wallet generator. Also as it would be the first result in searching "Bitcoin Paper Wallet" in google, we could assume the new user would go with this site.\\

\paragraph{CT1} On entering the site, The user is being asked to move the mouse or enter some random characters in the text box to make the random value to generate the bitcoin address. When the randomness is enough it will automatically redirects to the main page that shows the receiving bitcoin address and it's paired private key. The public key (Bitcoin address) is labeled by a green "SHARE" text and the private key by a red "SECRET" text that has to be kept secret. By clicking on "Print" Button the paper wallet is printed and is finalized. There is a short but complete documentation of the steps user has to take to achieve his desired functionality. So far user has his own bitcoin receiving address and as it is mentioned in the documentation to check his balance he has to go to block explorer sites such as blockchain.info and search for his Bitcoin Address. Although it has been mentioned to keep the private key secret, there might be users whom do not fully read the documentations and might expose the private key to public \footnote{A Bloomberg TV Host Gifted Bitcoin On Air And It Immediately Got Stolen\url{http://www.businessinsider.com/bloomberg-matt-miller-bitcoin-gift-stolen-2013-12}} (violates G5). The terminology and interface used in this site is really simple and user-friendly(complies G6).

\paragraph{CT2, CT3} For this technique, as the keys are printed on a paper, there is no difference between primary and secondary device.
To send funds from a bitcoin address that has been stored on a paper wallet, as it is mentioned in the documentation, user has to import his private key in one of the wallet clients available such as Armory(\ref{air gap}) blockchain.info online wallet (\ref{hosted}). For instance, on blockchain.info wallet, user can input the private key address and use the funds, however in this case the changes of the funds might be sent to his other addresses included in his online wallet and he might need to make another bitcoin paper wallet to send the remaining funds to. To do so, after making an online account in blockchain.info (\ref{hosted}), user could go to "Import/Export" tab and there is an option "Import Using Paper Wallet, Use your Webcam to scan a QR code from a paper wallet" that he can scan his paper wallet with the webcam and import the address into his online wallet. It is also possible to type in the private key in the "Import Private Key" text field. After this step, the address now is hosted on the online wallet and is the same as CT2 in Hosted wallet client (\ref{hosted transaction}).

\paragraph{CT4} The printed copy of the private key is the only way to hold on to the funds stored in the paired address. User has to acknowledge this (also mentioned in the short documentation), and save a copy of his keys in a safe place such as a personal vault, to be able to recover the funds in case the first copy is destroyed. In case of theft, there is no such a way to recover the keys.

%===========================================================

\subsubsection{Password-drived key} We would use the implementation of brain wallet in brainwallet.org, as it is the most popular and complete implementation on the time of writing.

\paragraph{CT1} On entering the site, There is an already generated address with the passphrase "correct horse battery staple". It should be mentioned that this is the default value and should not be used for personal use as the private key is publicly available\footnote{How to steal bitcoins \url{http://www.palkeo.com/code/stealing-bitcoin.html}}, this is a violation of G5. Now user has to type in his own passphrase and make sure his passphrase is not a commonly used phrase or anything that could be brute forced by a dictionary attack \footnote{Bitcoin “Brainwallets” and why they are a bad idea\url{http://insecurety.net/?p=866}} as this passphrase is everything that is needed to access the funds stored in the relevant bitcoin address. On entering the desired passphrase, the public and private keys are displayed in the same page. Now user has his own bitcoin address, and to check his balance, same as Paper Wallets, user has to go to a block explorer service and searches for his bitcoin address (nothing is mentioned in the site and this is a violation of G1,G2). There are lots of fields displayed that is not meaningful for the new user and might be confusing (violates G6 and G7).

\paragraph{CT2, CT3} In any device with a browser, User can browse to the brain wallet site and enter his passphrase to recover his keys. Although to authorize a transaction, user should import the private keys into another wallet client to be able to spend the funds, the procedure is the same as it was used to import paper wallets.

\paragraph{CT4} As long as user knows his passphrase, the keys are safe. That being said, one of the cons of using brainwallet is that if the site stops its service, the algorithm used to derive the keys from the passphrase is no longer available, in this case the best practice is to save the private key in a secure storage to be able to access it in case the site no longer worked.

%===========================================================

\subsubsection{Hosted}
\label{hosted}
There are different online services that offer online hosted wallet clients to users such as blockchain.info, coinbase \footnote{\url{https://coinbase.com/}}, same as previous sections we would work on todays popular online wallet, blockchain.info wallet.

\paragraph{CT1} User needs to go to the site\footnote{\url{https://blockchain.info/wallet}} and start a new wallet with his email address and a minimum of 10 character long password. There is a warning on the bottom of the form that on password loss the account is unrecoverable (Good use for G5), However the first pop up window titled "Wallet Recovery Mnemonic" would show a phrase that can be used to recover the account in case the password is forgotten. After this step user can login to the site and view his bitcoin address and the balance it holds.

\paragraph{CT2, CT3}
\label{hosted transaction}
One of the advantages of using hosted wallets is that it is accessible from any browser on any device. User can go to the site, log in and he has access to his funds. To do a transaction, user would click on "Send Money" tab of the main page, and there is a simple easy to use interface that asks for the destination bitcoin address and the amount of Bitcoins to be sent. Although in case of errors, errors might use a terminology that is unfamiliar for the user such as for insufficient funds this error that would be shown is "No free outputs to spend" (violates G4 and G6).

\paragraph{CT4} There are two methods of recovering, one is to use the phrase given on account registration, and the other is to use the backups. For the wallet recovery mnemonic, on the login page, there are different options for anything that is forgotten (identifier or alias for username and/or password). By clicking on the "Recover Wallet" option, it will ask for the mnemonic phrase and the email address or mobile phone number associated with the account to send the new confirmation details. The better recovery option is to have backups and import them in case recovery is needed. To do so, in the main wallet page, user has to click on the "Import/Export" option and Export either encrypted or unencrypted backup. It should be noted that unencrypted backup do not need a password to be read and should be kept in a secure and safe storage. There are different options for the unencrypted backup procedure that could confuse user and might result in unrecoverable back ups (Violates G5 and G6), also the back up would be shown on a text field that the user has to copy and paste in a text file to be able to save it on his computer (Violates G2, G3 and G7). To restore the backups, User need to go to Import Wallet option and copy and paste the backup file in the text field, in case of having the encrypted backup, the previously used password is also needed. 
