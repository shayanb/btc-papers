% !TEX root = ../main.tex

\section{Evaluation Methodology}

%intro from TOR paper -> undone citations
Another method to evaluate the usability of these techniques is heuristic evaluation (citation). In this paper, we employ cognitive walkthrough as our methodology for usability evaluation.
... 

% = = = = = = = = = = = = = = = = = = = = = = = = = = = = = = = = = = = = = = = = = =

\subsection {The Core Tasks}
The core tasks that we are going to perform for each key management technique are as follows.

\begin{itemize}
\renewcommand{\labelitemi}{\bf CT-1}
\item Finalize a receiving address from the primary device\footnote{primary device is the initial device that the key is generated on} 
\renewcommand{\labelitemi}{\bf CT-2}
\item Authorize the transaction from the primary device 
\renewcommand{\labelitemi}{\bf CT-3}
\item Authorize the transaction from the secondary device\footnote{any device other than the primary device}
\renewcommand{\labelitemi}{\bf CT-4}
\item Losing the main credential (recovery options)
\end{itemize}

The core tasks will be performed with the following clients with default configuration on each category:

\begin{enumerate}
\item Key in file (default):  \bitcoinclient 
\item Password Protected: \multibit 
\item Air gap: Bitcoin Armory \footnote{\url{https://bitcoinarmory.com/}}
\item Offline Storage: Paper Wallet
\item Password-drived key: Brain Wallet
\item Hosted: Blockchain.info wallet

\end{enumerate}

% = = = = = = = = = = = = = = = = = = = = = = = = = = = = = = = = = = = = = = = = = =

\subsubsection{Key in file}
The first step is to download and run \bitcoinclient, as it was described in section 4.1, it would take days to sync with current block chain, that being said, core task one that is the receiving address is available right after the first run in the "Receive" tab of the application. It is possible to send any owned funds from the "Send" tab of the application, also If the user clicks on his bitcoin address the "sign message" button would be enabled that the user could sign a custom message or just to show that he owns the address, this would be the CT2 as authorizing the transaction. CT-3 is not possible with \bitcoinclient as the new installation is needed on the new device thus new address is generated. As for CT4, if the user loses \walletfile, there is no way to recover the lost keys.


\subsubsection{Password Protected}
Analysis of the core tasks for the this section was done on \multibit that one of the most used bitcoin clients among newer users due to usage of SPV protocol and less time to sync the blockchain to show the updated amount of funds. CT1 to CT3 are the same as above technique, however for CT4, whether user has forgot his passphrase on the encrypted wallet or loses his wallet file, it is unlikely to recover the keys.


\subsubsection{Air gap}


\subsubsection{Offline Storage}


\subsubsection{Password-drived key}


\subsubsection{Hosted}
