% !TEX root = ../main.tex

\section{Results}

The following is the full details of our walkthroughs, which expands on the shorter version presented in the conference version of this paper.

\subsection{Keys in Local Storage (\bitcoinclient)}
We begin with an evaluation of \bitcoinclient, the original Bitcoin wallet client, which uses locally-stored keys. We assume the user has downloaded and installed the \bitcoinclient client (it has a straight forward wizard installation procedure).

\paragraph{T1: Configure} \bitcoinclient transparently generates a new set of addresses on first run, but shows no notification to the user that this has occurred (fails G3). The receiving address can be found under the \emph{Receive coins} tab, but this could be easily confused with the \emph{Addresses} tab which contains a contact list of other user addresses (fails G2). 

To retrieve the account balance, \bitcoinclient must be online and the user must wait until a full copy of the blockchain has been downloaded.  Except for a small status indicator on the bottom-right side of the window that shows a small red cross in-between two black windows, there are no other messages to show the user that the application should be online. Due to the size of the blockchain, this may take hours to days to complete. A status bar displaying ``Synchronizing with network'' shows the progress of the blockchain download (achieves G8), but the terminology may be too technical for novice users, With a mouse over the icon, it says `0 active connections to bitcoin network' which is likely unfamiliar language that does not help resolve the error (fails G4 and G6). Once the blockchain has been downloaded, the balance is displayed on the \emph{Overview} tab (achieves G3). 

\paragraph{T2: Spend} Spending Bitcoin is straightforward since the keys are readily available to the \bitcoinclient client. Users spend Bitcoin by navigating to the \emph{Spend} tab (achieves G1 and G2). Since our focus is on key management, we do not evaluate the actual completion of transactions (which may have additional usability issues). We focus on ensuring the key is available to the software tool (which is not so straightforward with \eg offline storage). 

\paragraph{T3: Spend from Secondary Device} Installing \bitcoinclient on a secondary device creates a new set of keys. Users may not understand that the keys must be copied to the secondary device (fails G1), and if so, what file must be copied (fails G2). The correct procedure is to back up the \walletfile with the `backup wallet...' option in the `File' tab of the first installation and chose a directory to save the \walletfile. Next the user must securely transfer this file to the secondary device, and no guidance is provided on how to do this (fails G2) or the dangers of transferring it through an insecure mechanism (fails G5).

Assuming the user has transferred \walletfile to the secondary device, she could try looking for import options in the newly installed wallet client, or drag and drop the \walletfile into the client, but she would fail to do so as no import option exists. The documentation is inadequate here as well---there is actually nothing in the help menu except a debug window that is for advance user to tweak the application (fails G2 and G6)!

The only mechanism to activate the wallet on a secondary file is to actually overwrite \walletfile on the secondary device with \walletfile from the first. It is unlikely any novice user would be able to complete this step. It is actually even difficult to find the path to copy \walletfile to on the new device---this could be possible by searching the local file system for \walletfile, which might not succeed due to non-searchable system reserved folders or not knowing the exact file name (spotlight does not return any result for \walletfile). More likely, the user will search online.\footnote{\url{https://en.bitcoin.it/wiki/Data_directory}} On OS X, the path is \texttt{/Users/User/Library/Application Support/Bitcoin/wallet.dat}.

The next step is to replace the new \walletfile with the one from the primary device. It should be noted that the name of the file should be exactly \walletfile for the \bitcoinclient to be able to read the file. Some of errors that the user might encounter during this procedure are:
\begin{itemize}
	\item The user might copy \walletfile from the primary device wallet client path instead of the one exported through the back up option. This could cause a corrupted \walletfile that is not readable by the secondary device's \bitcoinclient. This is due to \bitcoinclient's procedure to lock \walletfile while it is in use. The error is recoverable by repeating the process correctly (fails G4). 
	\item User should wait for the \bitcoinclient on the secondary device, to download and sync the Blockchain from the P2P network to be able to authorize a transaction.
	\item On the secondary device, the final balance might be wrong and there would be the need to resynchronize and rescan the blockchain to have the correct final balance (fails G3).
\end{itemize}

Finally, this process must be repeated if either client exhausts their keypool. If both do, there is no way to merge the new keys in the keypool, and replacing one \walletfile with the other will lead to unrecoverable funds (fails G5). 

We note that replacing the key file may require  a re-scan of the blockchain to display the correct balance (fails G3).

\paragraph{T4: Recovery} If only one device is used, there is no way to recover from loss of the key file (\eg due to a disk failure, file corruption, or loss of the device itself; fails G5). If the user backed-up the key file, the process for recovering from loss is equivalent to that of T3 above. 

%===========================================================

\subsection{Password-protected Wallets (\multibit)}
Although it is possible to encrypt the \walletfile with a password in \bitcoinclient, it is not the default option nor is there any cue to do so. Instead we evaluate the \multibit client, where one of the recommended first steps is to password protect the wallet file. \multibit is a popular client in particular for its use of SPV\footnote{Simplified Payment Verification\ref{Nak08}} for lightweight blockchain validation that can complete within minutes instead of, relative to \bitcoinclient, days.

\paragraph{T1: Configure} On first run, a welcome page contains an explanation of common tasks that can be performed with \multibit---where the send, request and transaction tabs are and how to password protect the wallet file (achieves G1 and G2). The client provides help options for other functionalities with direct and non-technical guides (achieves G6). 

\multibit automatically generates a new receiving address on first run, but does not notify the user (fails G3). Reading the newly generated address requires navigation to the \emph{Request} tab, which displays ``Your address'' as well as a copy of the QR code of address.(partially achieves G2).

The interface shows the status of the program (online, offline, or out of sync) on the bottom left status bar, the balance of the user's wallet on the upper left, and the latest price of bitcoin on the upper right of the window (achieves G8). The interface seems to minimize jargon and technical vocabulary (achieves G6). As it is mentioned on the \emph{welcome page} of the application, every option in \multibit has the ability to show help tips by hovering the mouse over that option (achieves G6 and G7).

The displayed balance is not necessarily current until synchronization is completed, however there is no direct cue in the balance area indicating this (achieves G8).

\paragraph{T2: Spend} The user must navigate to the tab labeled \emph{Send}, as instructed on the welcome screen (achieves G1 and G2). If the client is not synced, the send button is disabled (achieves G4). If it is synced, the user fills out the transaction details, destination address and amount and clicks send. The client prompts the user for the decryption password (achieves G2). An incorrect password displays the error `The wallet password is incorrect' but otherwise allows immediate and unlimited additional attempts. Entering the correct password authorizes the transaction (achieves G3).

% =JC= Upon the transaction being authorize, how is this communicated to the user? %NOTE: a pop up window with the details of the transaction(amount, recipient, ...) and some technical details (e.g how many peers have seen the broadcast)

\paragraph{T3: Spend from Secondary Device} On the primary device, the user must navigate to the \emph{Options} menu, and select \emph{Export private keys} under tools (fails G1 and G2). The interface displays a wizard requesting an export password as well as a file system path for the exported file to be saved. 
If the user attempts to save the exported file without password, a warning is displayed in red: `Anyone who can read your export file can spend your bitcoin' (achieves G5). By having a password-protected file exported, the user can securely copy the file to the secondary device with some protection against interception. After clicking to export, wallet file is saved in the given path and the client checks that the file is readable (achieves G4 and G5). 

On the secondary device, the user must select \emph{Import private keys} from the \emph{Options} menu. After selecting the previously exported file, the wizard confirms the completion of the import (achieves G4) and the balance is updated to reflect the newly imported keys. The user can proceed to create a new transaction as in T2. \multibit sends change to the originating address, so keypool churn is not an issue.

% =JC= "details about the current wallet files" -> be more specific, what details? %NOTE: Name, number of addresses in the wallet, path of the wallet file (e.g ~/../multibit.key)
% =JC= Can the wallet file have any name? %NOTE: Yes you can have multiple wallets with different names, changable from the interface 

\paragraph{T4: Recovery}.
As with \bitcoinclient, recovery is not possible if no backup of the wallet file was made. Creating a backup and importing it follows the same procedure as T3. As expected, both the password and the backed up wallet are necessary for recovery.

\subsection{Air-gapped Key Storage (\armory)}
\label{air gap}
Bitcoin \armory is an advanced bitcoin wallet that allows the wallet to be stored and managed on an offline device, while supporting the execution of a transaction through an online device. \armory is also used on the online device to obtain the blockchain and broadcast the transaction created on the offline device. It is possible to use some other online application to implement the airgap, however this is the recommend method and the one we will consider.

\paragraph{T1: Configure} The user begins by installing Bitcoin Armory on the offline computer. On the start, the welcome page offers the option to `Import Existing Wallet' and `Create Your First Wallet!' (achieves G2). The user creates her wallet, with passphrase-protection being a mandatory option. Armory asks to verify the passphrase and warns the user not to forget her passphrase (achieves G5). After this step, a backup window pops up with the options to print a paper wallet or save a digital backup of the wallet, and also warns the user if he decides not to backup his wallet (achieves G5). After proceeding, the user must click on `Receive Bitcoins' to prompt the client to generate the bitcoin address in the wallet file (fails G3,G4). By contrast, most clients do this step automatically on launch. 

In order to see the balance of the account, the device must be online and synced (fails G2). Thus the user must use the online device, not the offline device, to check her balance. Users can click on the `Offline Transaction' button, which offers a short documentation of the steps to be taken to sign a transaction and in doing so, explains the offline/online distinction relevant to checking a balance. Within the `Wallet' window, there is an option to `Create Watching-Only Copy.' The language is difficult (fails G6): this option allows a copy of only the bitcoin addresses to be exported, not the private keys, for use on the online computer to display an updated balance for each address. The exported file can be copied to the online computer.

We assume the user has installed \armory on the online computer. It should be noted that Armory only works side-by-side with \bitcoinclient and uses \bitcoinclient to synchronize and read the downloaded blockchain (fails G1). A pop-up window will alert the user when `the blockchain'  has been downloaded (partially achieves G3). \armory displays a `Connected' cue in green in the bottom-right when it connects to \bitcoinclient. Upon launching \armory, the user should click on `Import Existing Wallet' and she is prompted to import either a digital backup or watch-only wallet. She should chose the watch-only back up file that has been copied from the offline computer. After the application is done syncing, the balance is displayed on the main window under `available wallets' (achieves G8).

\textbf{T2 \& T3: Spend}.
With an air gap, the distinction between primary and secondary devices is less clear given that the basic setup itself includes two devices: one online and one offline, but authorization of transactions uses the offline device. To authorize a transaction, the user may begin from \armory on the online or offline device (may not fully achieve G2). On the either device, the user should click on `Offline Transactions' in the main window which displays a very detailed description of the steps involved (achieves G1, G2, and G6). On the online computer, the user clicks the option: \emph{Create New Offline Transaction}. The user will be asked to enter the transaction details to generate an unsigned transaction as a file. The user must transfer this file to the offline computer. As mentioned in this step's documentation, the unsigned transaction data has no private data (the exact data will ultimately be added to the public blockchain) and no harm can be done by an attacker who captures this file (achieves G5) other than learning the transaction is being prepared.

On the offline computer, the user clicks on \textit{Offline Transactions} and then \emph{Sign Offline Transaction} which prompts the user for the unsigned transaction data file. \armory asks the user to review all the transaction information, such as the amount and the receiving addresses (achieves G5). By clicking on the \textit{sign} button signed transaction data can be saved to a file. Text at the top of the window describes the current state of the file (signed) and what must be done (move to online device) to complete the transaction (achieves G1 and G2).

The signed file should be transferred to the online computer and be loaded through the same offline transaction window. When a signed transaction is detected, the \emph{Broadcast} button becomes clickable. By clicking on broadcast, the user can once more review transaction details, and receive confirmation that the Bitcoins have been sent (achieves G3 and G8).

\paragraph{T4: Recovery} Like \bitcoinclient and \multibit, \armory requires a backup of the wallet to have been made. Without this backup, recovery is impossible. \armory encourages backups at many stages (achieves G1 and G2). \armory provides many prompts for the user to back up her wallet keys. At the time of creating the wallet, there are multiple windows and alerts conveying the importance of a back up, with options for digital and paper copies. Even if user decides not to back up her wallet at this stage, she is provided a persistent `Backup This Wallet' option (achieves G4). In the backup window, there are a number of options to back up: a digital copy, paper copy, and others. By clicking on the `Make Paper Backup' for example, the paper backup is shown to the user containing a root key that consist of 18 sets of 4-characters words and a QR code. To restore the paper wallet backup, on the main page, the user can click on `Import or Restore Wallet' and select `Single-sheet Backup' option. She will be prompted to input the Root Key from the paper wallet. The `Digital Backup' option provides an unencrypted version of the wallet file that can be securely stored on portable media. Recovering from a lost wallet with a digital backup involves selecting `Import Digital Backup or watch-only wallet' from the `Import or Restore Wallet' window as explained in core task 1. \armory also enables the user to test the backups to ensure there is no error in the backup file (achieves G5) through the same import procedure.

%===========================================================
\subsection{Offline Storage (\paper)}
There are different methods to use for offline storage of a bitcoin wallet. For our evaluation, we consider the use of a paper wallet. Specifically we use the \paper web-service, a popular Bitcoin paper wallet generator. Many paper wallet generators exist, however \paper, at the time of writing, is the first search result for `bitcoin paper wallet' on Google.

\paragraph{T1: Configure} Upon visiting the \url{bitaddress.org}, the user is asked to move the mouse or enter random characters in a text box to generate a high-entropy random seed to be used to generate a private key associated with the Bitcoin address (achieves G1 and G2). Once enough entropy has been collected, the site redirects the user to a page that shows the receiving Bitcoin address and it's associated private key (achieves G3). The public key (Bitcoin address) is labeled \emph{Share} in green text and the private key \emph{Secret} in red (helping achieve G5). In general \paper uses non-expert terminology and simple instructions (achieves G6). To ensure the web service does not retain a copy of the users' key (the generation is done client-side in Javascript), the user should complete the process offline.

After printing, the user has a bitcoin receiving address and, as mentioned in the documentation, the balance can be checked through a Bitcoin Block Explorer\footnote{webservice that provides access to the blockchain}, such as \url{blockchain.info}. The user uses this site to search for her bitcoin address and checks her balance. Although it is documented that the private key must remain secret, a user may inadvertently expose the private key by placing the paper wallet where it can be observed or by searching the website for the private key instead of the public key. %NOTE: G5 or not G5? 

\paragraph{T2 \& T3: Spend} Since the keys are printed on paper, there is no difference between authorizing from a primary or secondary device so we collapse the analysis of core tasks 2 and 3.
 
To send funds from a bitcoin address that has been stored on a paper wallet, as it is mentioned in the documentation, the user has to import her private key in one of the wallet clients available, such as \armory or the \block hosted wallet discussed below. If the user inputs the private key address into a client that returns change to newly generated addresses, she must export these new addresses to a new paper wallet or she will lose the surplus when she removes the wallet from the client (fails G5). If the user fails to remove the wallet from the client, a second copy is maintained increasing her exposure to theft (but reducing her exposure to key loss). 

The process to import a key from a paper wallet depends on the client. For \block, after making an online account, the user navigates to the `Import/Export' tab and uses the option `Import Using Paper Wallet, Use your Webcam to scan a QR code from a paper wallet.' It is also possible to type in the private key in the `Import Private Key' text field. After this step, the address now is hosted on the online wallet and is the same as core task 2 in the Hosted Wallet section (Section~\ref{hosted transaction}) below.


\paragraph{T4: Recovery} Loss of a paper wallet makes the funds unrecoverable (fails G5). \paper prompts the user to acknowledge this fact (also mentioned in its short documentation) when creating a paper wallet (achieves G1).

%===========================================================
\subsection{Password-Derived Keys (\brain)} 
The most popular and complete implementation of a deterministic wallet with password-derived keys, at the time of writing, is \brain.

\paragraph{T1: Configure} The \brain website displays by default a pre-generated address corresponding to an empty passphrase. The passphrase input field displays ``Long original sentence that does not appear in any song or literature. Never use empty passphrase. (SHA256)'', but there is no corresponding documentation explaining the purpose of the passphrase or how it relates to the generated key (fails G1, G2, G6). As characters are typed, a new key is generated. User may not notice that generation of keys is happening dynamically, possibly preventing the user from noticing that the task is complete (fails G3). The user should replace the default passphrase with her own, hopefully ensuring her passphrase is not a commonly used phrase or anything that could be brute forced by an offline dictionary attack\footnote{``Bitcoin Brainwallets and why they are a bad idea'', \textit{Insecurety Research} (blog), 3/26/2013.} as this passphrase is sufficient to access the funds stored in the resulting bitcoin address. On entering the desired passphrase, the public and private keys are displayed on the same page.

Once the address has been generated, retrieving the balance of that address requires the use of an external service, but no suggestions are provided on the site (fails G1 and G2). The interface does display a number of other fields (\eg additional encodings of the public key) which may not be meaningful to novice users (fails G6 and G7). 

\paragraph{T2 \& T3: Spend} Spending Bitcoins from a password-derived wallet requires the user to import the private key into another client. The user should experience similar usability challenges as those detailed in the Offline Storage client above.

\paragraph{T4: Recovery} Forgetting the password of a password-derived key leads to funds becoming unrecoverable (fails G5). Users will typically return to the same website (\ie the \brain website) to extract private keys, but this may not be possible if the site is inaccessible (fails G5). 

%===========================================================
\subsection{Hosted Wallets (\block)}
\label{hosted}
A variety of online services offer online hosted wallet clients to users. We use the popular \block webservice for our evaluation.

\paragraph{T1: Configure} The user navigates to the \block site and creates a new wallet by providing an email address and a (min) 10 character password (achieves G1 and G2). A message warning the user about the importance of not forgetting the password is displayed during registration (achieves G5). Next, a \emph{Wallet Recovery Mnemonic} is shown to the user as a backup in case the password is forgotten. The balance and address are immediately displayed (achieves G3). 

\paragraph{T2 \& T3: Spend}
\label{hosted transaction} 
Hosted wallets are accessible from any web browser, so creating transactions from many devices is straightforward. The user logs in to the site, clicks \emph{Send money} (achieves G1 and G2). After filling in the required fields, the user is informed that the Bitcoins have been sent (achieves G3). Some of \block 's error messages may be too technical for novice users. For example, \emph{No free outputs to spend} is displayed when transactions are created without sufficient funds (fails G6). 

\paragraph{T4: Recovery} To recover from a forgotten password, a wallet recovery mnemonic may be provided on the login page. By clicking the \emph{Recover Wallet} button, the site will ask for the mnemonic phrase and the email address send the new credentials (achieves G1 and G2). Another recovery option is to proactively make backups and import them in case recovery is needed. To do so, in the main wallet page, user has to click \emph{Import/Export} and exporting either an encrypted or unencrypted backup. Unencrypted backups should be kept in a secure storage. There are different options for the unencrypted backup procedure that could confuse the user and might result in an unrecoverable backup (fails G5 and G6)---the back up is shown on a text field that the user has to copy and paste into a text file to be able to save it on her computer (fails G2, G3 and G7). To restore the backups, there is an `Import Wallet' option.
% = = = = = = = = = = = = = = = = = = = = = = = = = = = = = = = = = = = = = = = = = =
